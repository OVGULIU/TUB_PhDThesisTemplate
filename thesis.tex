% ----------------------------------------------------------------------
%                   LATEX TEMPLATE FOR PhD THESIS
% ----------------------------------------------------------------------

% based on Harish Bhanderi's PhD/MPhil template, then Uni Cambridge
% http://www-h.eng.cam.ac.uk/help/tpl/textprocessing/ThesisStyle/
% corrected and extended in 2007 by Jakob Suckale, then MPI-CBG PhD programme
% and made available through OpenWetWare.org - the free biology wiki
% and finally modified in 2015-2017 by Holger Nahrstaedt
% https://github.com/holgern/TUB_PhDThesisTemplate

%: Style file for Latex
% Most style definitions are in the external file PhDthesisPSnPDF.
% In this template package, it can be found in ./Classes/

\documentclass[twoside,11pt,online,a4paper,libertine,pdfa1,custommargin]{Classes/PhDthesisPSnPDF}
% *********************** Choosing pdfx standard ******************************
% `pdfa1'
% `pdfa2'
% `pdfx3'
%
% % *********************** Choosing oneside / twoside ******************************
% `oneside' : layout is optimized for one-side print
% `twoside' : layout is optimized for two-side print
% *********************** Choosing print / online ******************************
% `print' : pdf-file is optimized for print
% `online' : pdf-file is optimized for online submission. The links are colorfull.
% *********************** Choosing biblatex or bibtex ******************
% `biblatex' : biblatex is used. Biblatex is automatically set when using xetex
%
% % *********************** Choosing the Fonts size ******************
% `9pt'
% `10pt'
% `11pt'
% `12pt'
% % *********************** Choosing the paper size ******************
% `letterpaper'
% `a4paper'
% `a5paper'
% *********************** Choosing the Fonts in Class Options ******************
%
% `times' : Times font with math support. (The Cambridge University guidelines
% recommend using times)
%
% `fourier': Utopia Font with Fourier Math font (Font has to be installed)
%            It's a free font.
% 'libertine' : Libertine Font with Math fonts (newtxmath)
%
% `customfont': Use `customfont' option in the document class and load the
% package in the preamble.tex
%
% default or leave empty: `Latin Modern' font will be loaded.
%
% ************************* Custom Page Margins ********************************
%
% `custommargin`: Use `custommargin' in options to activate custom page margins,
% which can be defined in the preamble.tex. Custom margin will override
% print/online margin setup.
%
% ************************* other options ********************************
% `abstract`: Only the title-page and the abstracts are generated
%

% -*- root: ../thesis.tex -*-
%!TEX root = ../thesis.tex
% ******************************************************************************
% ****************************** Custom Margin *********************************
% Add `custommargin' in the document class options to use this section
% Set {innerside margin / outerside margin / topmargin / bottom margin}  and
% other page dimensions
\ifCLASSINFOcustommargin
  %\RequirePackage[left=37mm,right=30mm,top=35mm,bottom=30mm]{geometry}
  \RequirePackage[left=32mm,right=22mm,top=12mm,bottom=10mm,includeheadfoot,heightrounded]{geometry}

%\setlength\marginparwidth{2.3cm} %Die wird später zum Rechnen gebraucht, wird aber durch die Angabe im geometry package nicht automatisch richtig gesetzt.
  \setFancyHdr % To apply fancy header after geometry package is loaded
\fi
%\overfullrule=5pt

%: ----------------------------------------------------------------------
%:                  TITLE PAGE: name, degree,..
% ----------------------------------------------------------------------
% below is to generate the title page with crest and author name

%if output to PDF then put the following in PDF header
\ifpdf  
  \ifCLASSINFOpdfxthree
  \else
      \pdfcatalog { /PageMode (/UseOutlines)
                    /OpenAction (fitbh)  }
  \fi
\fi
%\usepackage{showframe}
\ifCLASSINFObiblatex
\usepackage[
    backend=biber,
    style=ieee,
    sortlocale=en_US,
    natbib=true,
    maxbibnames=50,
    url=false, 
    doi=true,
    eprint=false
]{biblatex}
\addbibresource{9_backmatter/references.bib}
\DeclareSourcemap{ 
    \maps[datatype=bibtex]{
      \map{
           \step[fieldsource=doi, match={\regexp{\{\\textunderscore.?\}}}, replace={_}]
           \step[fieldsource=doi, match={\regexp{\{\\textless.?\}}}, replace={&lt;}]
           \step[fieldsource=doi, match={\regexp{\{\\textgreater.?\}}}, replace={&gt;}]
           \step[fieldsource=doi, match={\regexp{\{\>.?\}}}, replace={&gt;}]
      }
      %\map{
      %     \step[fieldsource=doi, match={\regexp{\{\\textless.?\}}}, replace={<}]
      %     %\step[fieldsource=doi, match={\regexp{\{\\textgreater.*\}}}, replace={>}]
      %}
      %\map{
      %     \step[fieldsource=doi, match={\regexp{\{\\textgreater *\}}}, replace={>}]
      %     %\step[fieldsource=doi, match={\regexp{\{\\textgreater.*\}}}, replace={>}]
      %}
    }
}
\else
\usepackage[sort, numbers]{natbib}
\fi

% ----------------------------------------------------------------------
       
% turn of those nasty overfull and underfull hboxes

%\hbadness=10000
%\hfuzz=50pt

\tolerance=1414
\hbadness=1414
\emergencystretch=1.5em
\hfuzz=0.5pt
%\widowpenalty=10000
\vfuzz=\hfuzz
\raggedbottom

% TeX default is 50
\hyphenpenalty=750
% The TeX default is 1000
%\hbadness=1350
% IEEE does not use extra spacing after punctuation
\frenchspacing

\binoppenalty=1000 % default 700
\relpenalty=800     % default 500
   
\interfootnotelinepenalty=10000

% Don't break enumeration (etc.) across pages in an ugly manner
\clubpenalty=10000
\widowpenalty=10000

% ********************** TOC depth and numbering depth *************************
% levels are: 0 - chapter, 1 - section, 2 - subsection, 3 - subsection
\setcounter{secnumdepth}{3} % organisational level that receives a numbers
\setcounter{tocdepth}{3}    % print table of contents for level 3

% Special layout for chapter numbers
\titleformat{\chapter}[display]
{\bfseries\sffamily\Huge}
{\hfill\fontsize{140}{50}\selectfont\color{lightgray}\rmfamily\textbf{\thechapter}}% label
{-0ex}
%{\filleft moves all to the right side
{\filleft\fontsize{50}{50}}
[\vspace{-0ex}]

\title{Writing your thesis with LateX}
\subtitle{Using the TUB\_PhDThesisTemplate}



% ----------------------------------------------------------------------
% The section below defines www links/email for author and institutions
% They will appear on the title page of the PDF and can be clicked
\ifpdf
  % The crest is a graphics file of the logo of your research institution.
  % Place it in ./0_frontmatter/figures and specify the width
  \crest{}
% If you are not creating a PDF then use the following. The default is PDF.
\else
%  \crest{\includegraphics[width=4cm]{logo.png}}
  \crest{}
\fi
  \author{Max Mustermann}
%  \cityofbirth{born in XYZ} % uncomment this if your university requires this
  \cityofbirth{Berlin}
%  % If city of birth is required, also uncomment 2 sections in PhDthesisPSnPDF
%  % Just search for the "city" and you'll find them.
\collegeordept{von der Fakult\"at IV - Elektrotechnik und Informatik}
\university{der Technischen Universit\"at Berlin}
\degreefull{Doktor der Ingenieurwissenschaften}
\olddegree{Dipl.-Ing.}
\degree{-Dr.-Ing.-}
\degreedate{Tag der wissenschaftlichen Aussprache: XX. xxxx 2016}
\degreeplaceyear{Berlin 2016}
\comiteehead{-}
\firstreviewer{-}
\secondreviewer{-}
\thirdreviewer{-}
%\forthreviewer{-}

%: ----------------------- set languange ------------------------
\selectlanguage{english}

% ***************************** Abstract Separate ******************************
% To printout only the titlepage and the abstract with the PhD title and the
% author name for submission to the Student Registry, use the `abstract' option in
% the document class.

\ifdefineAbstract
 \pagestyle{empty}
 \includeonly{0_frontmatter/zusammenfassung, 0_frontmatter/abstract}
\fi

%: ----------------------- generate glossary ------------------------
\loadglsentries{0_frontmatter/glossary}
\makeglossaries
\begin{document}



%: ----------------------- generate cover page ------------------------
\frontmatter
% \maketitle generates a title page for the final submission
% \makepretitle generates a title page for evaluation process
\maketitle  
%\makepretitle

%: ----------------------- Choose spacing ------------------------
%\singlespacing
\onehalfspacing
%\doublespacing

%: ----------------------- abstract ------------------------

% Your institution may have specific regulations if you need an abstract and where it is to be placed in the document. The default here is just after title.
% -*- root: ../thesis.tex -*-
% Thesis Abstract -----------------------------------------------------
\selectlanguage{german}
\begin{zusammenfassung}        %this creates the heading for the abstract page
Hier kommt der deutsche Abstrakt rein...
ÜÖ sind ok.
\end{zusammenfassung}
\ifCLASSINFOlangDE
\selectlanguage{german}
\else
\selectlanguage{english}
\fi
% ---------------------------------------------------------------------- 


% Thesis Abstract -----------------------------------------------------
\ifCLASSINFOlangDE
\selectlanguage{english}
\fi

%\begin{abstractslong}    %uncommenting this line, gives a different abstract heading
\begin{abstracts}        %this creates the heading for the abstract page

Put your abstract here...

\end{abstracts}
%\end{abstractlongs}
\ifCLASSINFOlangDE
\selectlanguage{german}
\fi

% ---------------------------------------------------------------------- 



%: ----------------------- tie in front matter ------------------------

%\frontmatter
% -*- root: ../thesis.tex -*-
% Thesis Dedictation ---------------------------------------------------

\begin{dedication} %this creates the heading for the dedication page

Dedicated to ...

\end{dedication}

% ----------------------------------------------------------------------
% Thesis Acknowledgements ------------------------------------------------


%\begin{acknowledgementslong} %uncommenting this line, gives a different acknowledgements heading
\begin{acknowledgements}      %this creates the heading for the acknowlegments

I would like to acknowledge the thousands of individuals who have coded for open-source projects for free. It is due to their efforts that 
scientific work with powerful tools is possible.


\end{acknowledgements}
%\end{acknowledgmentslong}

% ------------------------------------------------------------------------





%: ----------------------- contents ------------------------
\tableofcontents            % print the table of contents

%: ----------------------- list of figures/tables ------------------------
\cleardoublepage
\listoffigures	% print list of figures
\cleardoublepage
\listoftables  % print list of tables


%: ----------------------- glossary ------------------------

% Tie in external source file for definitions: /0_frontmatter/glossary.tex
% Glossary entries can also be defined in the main text. See glossary.tex
\cleardoublepage
%\chapter{Glossary}
\begin{multicols}{2} % \begin{multicols}{#columns}[header text][space]
\begin{footnotesize} % scriptsize(7) < footnotesize(8) < small (9) < normal (10)
\printglossary[type=\acronymtype,title=Abbreviations]
%\printglossary
%\printnomenclature[1.5cm] % [] = distance between entry and description
%\printglossery
\label{nom} % target name for links to glossary
\end{footnotesize}
\end{multicols}

\begin{multicols}{2} % \begin{multicols}{#columns}[header text][space]
\begin{footnotesize} 
\printglossary[type=symbolslist,title=Symbols]
\end{footnotesize}
\end{multicols}
%: --------------------------------------------------------------
%:                  MAIN DOCUMENT SECTION
% --------------------------------------------------------------

\mainmatter

\renewcommand{\chaptername}{} % uncomment to print only "1" not "Chapter 1"

%: ----------------------- subdocuments ------------------------

% Parts of the thesis are included below. Rename the files as required.
% But take care that the paths match. You can also change the order of appearance by moving the include commands.
% \cfchapter[short name] {full name} {folder name} {file name}.
\tikzsetexternalprefix{./1_introduction/TikzPictures/}
\cfchapter{Introduction}{1_introduction}{introduction}
\tikzsetexternalprefix{./2/TikzPictures/}
\cfchapter{State of the Art}{2}{chapter2}
\tikzsetexternalprefix{./3/TikzPictures/}
\cfchapter{Including tikz\label{ch:chapter3}}{3}{chapter3}
\tikzsetexternalprefix{./4/TikzPictures/}
\cfchapter{Sum - Algorithm\label{ch:chapter4}}{4}{chapter4}
\tikzsetexternalprefix{./5/TikzPictures/}
\cfchapter{PGF-plots from python\label{ch:chapter5}}{5}{chapter5}
\tikzsetexternalprefix{./6/TikzPictures/}
\cfchapter{Asymptote\label{ch:chapter6}}{6}{chapter6}
\tikzsetexternalprefix{./7/TikzPictures/}
\cfchapter{Discussion\label{ch:chapter7}}{7}{discussion}
\tikzsetexternalprefix{./8/TikzPictures/}
\cfchapter{Materials and Methods}{8}{materials_methods}
\cleardoublepage
       % description of lab methods



% --------------------------------------------------------------
%:                  BACK MATTER: appendices, refs,..
% --------------------------------------------------------------

% the back matter: appendix and references close the thesis


%: ----------------------- bibliography ------------------------

% The section below defines how references are listed and formatted
% The default below is one column, small font, complete author names.
% Entries are also linked back to the page number in the text and to external URL if provided in the BibTex file.

\begin{footnotesize} % tiny(5) < scriptsize(7) < footnotesize(8) < small (9)


\renewcommand{\bibname}{References} % changes the header; default: Bibliography
\ifCLASSINFObiblatex
\printbibliography
\else
% PhDbiblio-url2 = names small caps, title bold & hyperlinked, link to page 
\bibliographystyle{Classes/PhDbiblio-url2} % Title is link if provided
\bibliography{9_backmatter/references} % adjust this to fit your BibTex file
\fi
\end{footnotesize}



% --------------------------------------------------------------


%: Declaration of originality
%
% Thesis statement of originality -------------------------------------

% Depending on the regulations of your faculty you may need a declaration like the one below. This specific one is from the medical faculty of the university of Dresden.

\begin{declaration}        %this creates the heading for the declaration page

I herewith declare that I have produced this paper without the prohibited assistance of third parties and without making use of aids other than those specified; notions taken over directly or indirectly from other sources have been identified as such. This paper has not previously been presented in identical or similar form to any other German or foreign examination board.

The thesis work was conducted from XXX to YYY under the supervision of PI at ZZZ.

\vspace{10mm}

CITY,


\end{declaration}


% ----------------------------------------------------------------------

% ********************************** Appendices ********************************

\begin{appendices} % Using appendices environment for more functionality
\tikzsetexternalprefix{./Appendix1/TikzPictures/}
%!TEX root = ../thesis.tex
% -*- root: ../thesis.tex -*-
% ******************************* Thesis Appendix A ****************************
\chapter{Appendix A } 

\end{appendices}

\end{document}
