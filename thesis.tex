% ----------------------------------------------------------------------
%                   LATEX TEMPLATE FOR PhD THESIS
% ----------------------------------------------------------------------

% based on Harish Bhanderi's PhD/MPhil template, then Uni Cambridge
% http://www-h.eng.cam.ac.uk/help/tpl/textprocessing/ThesisStyle/
% corrected and extended in 2007 by Jakob Suckale, then MPI-CBG PhD programme
% and made available through OpenWetWare.org - the free biology wiki
% and finally modified in 2015-2016 by Holger Nahrstaedt

%: Style file for Latex
% Most style definitions are in the external file PhDthesisPSnPDF.
% In this template package, it can be found in ./Latex/Classes/
\documentclass[twoside,12pt,online,a4paper,times,pdf1a]{Latex/Classes/PhDthesisPSnPDF}


% *********************** Choosing the Fonts in Class Options ******************
%
% `times' : Times font with math support. (The Cambridge University guidelines
% recommend using times)
%
% `fourier': Utopia Font with Fourier Math font (Font has to be installed)
%            It's a free font.
% 'libertine' : Libertine Font with Math fonts (newtxmath)
%
% `customfont': Use `customfont' option in the document class and load the
% package in the preamble.tex
%
% default or leave empty: `Latin Modern' font will be loaded.
%
% ************************* Custom Page Margins ********************************
%
% `custommargin`: Use `custommargin' in options to activate custom page margins,
% which can be defined in the preamble.tex. Custom margin will override
% print/online margin setup.
%

%: Macro file for Latex
% Macros help you summarise frequently repeated Latex commands.
% Here, they are placed in an external file /Latex/Macros/MacroFile1.tex
% An macro that you may use frequently is the figuremacro (see introduction.tex)
% This file contains macros that can be called up from connected TeX files
% It helps to summarise repeated code, e.g. figure insertion (see below).

% insert a centered figure with caption and description
% parameters 1:filename, 2:title, 3:description and label
\newcommand{\figuremacro}[3]{
	\begin{figure}[htbp]
		\centering
		\includegraphics[width=1\textwidth]{#1}
		\caption[#2]{\textbf{#2} - #3}
		\label{#1}
	\end{figure}
}

% insert a centered figure with caption and description AND WIDTH
% parameters 1:filename, 2:title, 3:description and label, 4: textwidth
% textwidth 1 means as text, 0.5 means half the width of the text
\newcommand{\figuremacroW}[4]{
	\begin{figure}[htbp]
		\centering
		\includegraphics[width=#4\textwidth]{#1}
		\caption[#2]{\textbf{#2} - #3}
		\label{#1}
	\end{figure}
}

% inserts a figure with wrapped around text; only suitable for NARROW figs
% o is for outside on a double paged document; others: l, r, i(inside)
% text and figure will each be half of the document width
% note: long captions often crash with adjacent content; take care
% in general: above 2 macro produce more reliable layout
\newcommand{\figuremacroN}[3]{
	\begin{wrapfigure}{o}{0.5\textwidth}
		\centering
		\includegraphics[width=0.48\textwidth]{#1}
		\caption[#2]{{\small\textbf{#2} - #3}}
		\label{#1}
	\end{wrapfigure}
}

% predefined commands by Harish
\newcommand{\PdfPsText}[2]{
  \ifpdf
     #1
  \else
     #2
  \fi
}

\newcommand{\IncludeGraphicsH}[3]{
  \PdfPsText{\includegraphics[height=#2]{#1}}{\includegraphics[bb = #3, height=#2]{#1}}
}

\newcommand{\IncludeGraphicsW}[3]{
  \PdfPsText{\includegraphics[width=#2]{#1}}{\includegraphics[bb = #3, width=#2]{#1}}
}


\newcommand{\IncludePdfTex}[2]{
 %\def\svgwidth{8.0cm}
 \def\svgwidth{#1\columnwidth}
  \input{#2}  
}

% sets line spacing
\renewcommand\baselinestretch{1.5}
%\renewcommand\baselinestretch{2}
%\renewcommand\baselinestretch{3}
\baselineskip=18pt plus1pt

\newcommand{\InsertFig}[3]{
  \begin{figure}[!htbp]
    \begin{center}
      \leavevmode
      #1
      \caption{#2}
      \label{#3}
    \end{center}
  \end{figure}
}


%%% Local Variables: 
%%% mode: latex
%%% TeX-master: "~/Documents/LaTeX/CUEDThesisPSnPDF/thesis"
%%% End: 


% Add `custommargin' in the document class options to use this section
% Set {innerside margin / outerside margin / topmargin / bottom margin}  and
% other page dimensions
\ifCLASSINFOcustommargin
  \RequirePackage[left=37mm,right=30mm,top=35mm,bottom=30mm]{geometry}
  %\setFancyHdr % To apply fancy header after geometry package is loaded
\fi

%: ----------------------------------------------------------------------
%:                  TITLE PAGE: name, degree,..
% ----------------------------------------------------------------------
% below is to generate the title page with crest and author name

%if output to PDF then put the following in PDF header
\ifpdf  
    \pdfcatalog { /PageMode (/UseOutlines)
                  /OpenAction (fitbh)  }
\fi


\title{Writing your thesis with LateX}


% ----------------------------------------------------------------------
% The section below defines www links/email for author and institutions
% They will appear on the title page of the PDF and can be clicked
\ifpdf
  % The crest is a graphics file of the logo of your research institution.
  % Place it in ./0_frontmatter/figures and specify the width
  \crest{}
% If you are not creating a PDF then use the following. The default is PDF.
\else
%  \crest{\includegraphics[width=4cm]{logo.png}}
  \crest{}
\fi
  \author{Holger Nahrstaedt}
%  \cityofbirth{born in XYZ} % uncomment this if your university requires this
  \cityofbirth{Berlin}
%  % If city of birth is required, also uncomment 2 sections in PhDthesisPSnPDF
%  % Just search for the "city" and you'll find them.
\collegeordept{von der Fakult\"at IV - Elektrotechnik und Informatik}
\university{der Technischen Universit\"at Berlin}
\degreefull{Doktor der Ingenieurwissenschaften}
\olddegree{Dipl.-Ing.}
\degree{-Dr.-Ing.-}
\degreedate{Tag der wissenschaftlichen Aussprache: XX. xxxx 2016}
\degreeplaceyear{Berlin 2016}
\comiteehead{-}
\firstreviewer{-}
\secondreviewer{-}
\thirdreviewer{-}



% ----------------------------------------------------------------------
       
% turn of those nasty overfull and underfull hboxes
\hbadness=10000
\hfuzz=50pt


%: --------------------------------------------------------------
%:                  FRONT MATTER: dedications, abstract,..
% --------------------------------------------------------------
\selectlanguage{english}
\loadglsentries{0_frontmatter/glossary}
\makeglossaries
\begin{document}

%\language{english}




%: ----------------------- generate cover page ------------------------
\frontmatter
\maketitle  % command to print the title page with above variables




\singlespacing
%\onehalfspacing
%\doublespacing

%: ----------------------- abstract ------------------------

% Your institution may have specific regulations if you need an abstract and where it is to be placed in the document. The default here is just after title.
\selectlanguage{german}
% -*- root: ../thesis.tex -*-
% Thesis Abstract -----------------------------------------------------
\selectlanguage{german}
\begin{zusammenfassung}        %this creates the heading for the abstract page
Hier kommt der deutsche Abstrakt rein...
ÜÖ sind ok.
\end{zusammenfassung}
\ifCLASSINFOlangDE
\selectlanguage{german}
\else
\selectlanguage{english}
\fi
% ---------------------------------------------------------------------- 

\selectlanguage{english}

% Thesis Abstract -----------------------------------------------------
\ifCLASSINFOlangDE
\selectlanguage{english}
\fi

%\begin{abstractslong}    %uncommenting this line, gives a different abstract heading
\begin{abstracts}        %this creates the heading for the abstract page

Put your abstract here...

\end{abstracts}
%\end{abstractlongs}
\ifCLASSINFOlangDE
\selectlanguage{german}
\fi

% ---------------------------------------------------------------------- 



%: ----------------------- tie in front matter ------------------------

%\frontmatter
% -*- root: ../thesis.tex -*-
% Thesis Dedictation ---------------------------------------------------

\begin{dedication} %this creates the heading for the dedication page

Dedicated to ...

\end{dedication}

% ----------------------------------------------------------------------
% Thesis Acknowledgements ------------------------------------------------


%\begin{acknowledgementslong} %uncommenting this line, gives a different acknowledgements heading
\begin{acknowledgements}      %this creates the heading for the acknowlegments

I would like to acknowledge the thousands of individuals who have coded for open-source projects for free. It is due to their efforts that 
scientific work with powerful tools is possible.


\end{acknowledgements}
%\end{acknowledgmentslong}

% ------------------------------------------------------------------------





%: ----------------------- contents ------------------------

\setcounter{secnumdepth}{3} % organisational level that receives a numbers
\setcounter{tocdepth}{3}    % print table of contents for level 3
\tableofcontents            % print the table of contents
% levels are: 0 - chapter, 1 - section, 2 - subsection, 3 - subsection


%: ----------------------- list of figures/tables ------------------------

\listoffigures	% print list of figures

\listoftables  % print list of tables


%: ----------------------- glossary ------------------------

% Tie in external source file for definitions: /0_frontmatter/glossary.tex
% Glossary entries can also be defined in the main text. See glossary.tex
\newpage
%\chapter{Glossary}
\begin{multicols}{2} % \begin{multicols}{#columns}[header text][space]
\begin{footnotesize} % scriptsize(7) < footnotesize(8) < small (9) < normal (10)
\printglossary[type=\acronymtype,title=Abbreviations]
%\printglossary
%\printnomenclature[1.5cm] % [] = distance between entry and description
%\printglossery
\label{nom} % target name for links to glossary
\end{footnotesize}
\end{multicols}

\begin{multicols}{2} % \begin{multicols}{#columns}[header text][space]
\begin{footnotesize} 
\printglossary[type=symbolslist,title=Symbols]
\end{footnotesize}
\end{multicols}
%: --------------------------------------------------------------
%:                  MAIN DOCUMENT SECTION
% --------------------------------------------------------------
%\newcommand{\useexternalfile}[1]{%
%    \tikzsetnextfilename{#1-output}%
%    \input{\tikzexternal@filenameprefix#1.tikz}}
% the main text starts here with the introduction, 1st chapter,...
\mainmatter

\renewcommand{\chaptername}{} % uncomment to print only "1" not "Chapter 1"

%: ----------------------- subdocuments ------------------------

% Parts of the thesis are included below. Rename the files as required.
% But take care that the paths match. You can also change the order of appearance by moving the include commands.
% \cfchapter[short name] {full name} {folder name} {file name}.
\tikzsetexternalprefix{./1_introduction/TikzPictures/}
\cfchapter{Introduction}{1_introduction}{introduction}
\tikzsetexternalprefix{./2/TikzPictures/}
\cfchapter{State of the Art}{2}{chapter2}
\tikzsetexternalprefix{./3/TikzPictures/}
\cfchapter{Including tikz\label{ch:chapter3}}{3}{chapter3}
\tikzsetexternalprefix{./4/TikzPictures/}
\cfchapter{chapter 4\label{ch:chapter4}}{4}{chapter4}
\tikzsetexternalprefix{./5/TikzPictures/}
\cfchapter{chapter 5\label{ch:chapter5}}{5}{chapter5}
\tikzsetexternalprefix{./5/TikzPictures/}
\cfchapter{Asymptote\label{ch:chapter6}}{6}{chapter6}
\tikzsetexternalprefix{./5/TikzPictures/}
\cfchapter{Discussion\label{ch:chapter7}}{7}{discussion}
\tikzsetexternalprefix{./8/TikzPictures/}
\cfchapter{Materials and Methods}{8}{materials_methods}
\newpage
       % description of lab methods




% --------------------------------------------------------------
%:                  BACK MATTER: appendices, refs,..
% --------------------------------------------------------------

% the back matter: appendix and references close the thesis


%: ----------------------- bibliography ------------------------

% The section below defines how references are listed and formatted
% The default below is 2 columns, small font, complete author names.
% Entries are also linked back to the page number in the text and to external URL if provided in the BibTex file.

% PhDbiblio-url2 = names small caps, title bold & hyperlinked, link to page 
\begin{multicols}{2} % \begin{multicols}{ # columns}[ header text][ space]
\begin{tiny} % tiny(5) < scriptsize(7) < footnotesize(8) < small (9)
\cleardoublepage
\bibliographystyle{Latex/Classes/PhDbiblio-url2} % Title is link if provided
\renewcommand{\bibname}{References} % changes the header; default: Bibliography

\bibliography{9_backmatter/references} % adjust this to fit your BibTex file

\end{tiny}
\end{multicols}

% --------------------------------------------------------------
% Various bibliography styles exit. Replace above style as desired.

% in-text refs: (1) (1; 2)
% ref list: alphabetical; author(s) in small caps; initials last name; page(s)
%\bibliographystyle{Latex/Classes/PhDbiblio-case} % title forced lower case
%\bibliographystyle{Latex/Classes/PhDbiblio-bold} % title as in bibtex but bold
%\bibliographystyle{Latex/Classes/PhDbiblio-url} % bold + www link if provided

%\bibliographystyle{Latex/Classes/jmb} % calls style file jmb.bst
% in-text refs: author (year) without brackets
% ref list: alphabetical; author(s) in normal font; last name, initials; page(s)

%\bibliographystyle{plainnat} % calls style file plainnat.bst
% in-text refs: author (year) without brackets
% (this works with package natbib)


% --------------------------------------------------------------


%: Declaration of originality
%
% Thesis statement of originality -------------------------------------

% Depending on the regulations of your faculty you may need a declaration like the one below. This specific one is from the medical faculty of the university of Dresden.

\begin{declaration}        %this creates the heading for the declaration page

I herewith declare that I have produced this paper without the prohibited assistance of third parties and without making use of aids other than those specified; notions taken over directly or indirectly from other sources have been identified as such. This paper has not previously been presented in identical or similar form to any other German or foreign examination board.

The thesis work was conducted from XXX to YYY under the supervision of PI at ZZZ.

\vspace{10mm}

CITY,


\end{declaration}


% ----------------------------------------------------------------------



\end{document}
